\chapter{Concepts}\label{cha:maltcms}

\section{Testing Framework}\label{sec:testing-framework}
Maltcms' testing infrastructure is based on \href{http://www.junit.org}{JUnit 4.10}. The \verb|maltcms-test-tools| folder below the main Maltcms 
Maven project contains two essential sub-projects: \verb|maltcms-test-defaults| and \verb|maltcms-test-data| that are built and distributed individually
from Maltcms. Both of these dependencies should be included on projects whishing to use the Maltcms testing infrastructure. The parent pom of Maltcms does 
this automatically for all its child projects.

\subsection{Plain Unit Testing}\label{sec:plain-unit-testing}
Simple unit tests in Maltcms are easy to define, following the \href{http://junit.sourceforge.net/doc/cookbook/cookbook.htm}{JUnit cookbook}. First,
create a new class below the \verb|src/test/java| folder of your Maven project. Then, for each behaviour of your class that you want to test, create 
one declaratively named method, optionally with a \verb|test| prefix to better distinguish it from auxiliary code. Add the \verb|@org.junit.Test| annotation
before or in the line above the method's signature. Within the method, implement your testing logic, using the usual \verb|Assert.assert*| methods. An example is given in listing \ref{lst:plain-unit-test}, assuming the class \verb|MyFirstClass| is located in \verb|src/main/java/my/first/module|.
Run \verb|mvn test| on your project and watch for the test summary output.

\begin{lstlisting}[caption={[Plain unit test]Example of a plain unit test},label=lst:plain-unit-test]
package my.first.module;

import org.junit.Test;
//static import since Java 1.5
import static org.junit.Assert.*;

public class MyUnitTest {	
	@Test
	public void testMyAssertion() {
		MyFirstClass mfc = new MyFirstClass();
		assertTrue(mfc.isReady());
	}
}
\end{lstlisting}

\subsection{Integration Testing}\label{sec:integration-testing}
The \href{http://maven.apache.org/plugins/maven-failsafe-plugin}{maven-failsafe-plugin} provides the execution logic for integration tests. If you want to define your unit test to be an integration test, simply 
annotate your class with \verb|@Category(IntegrationTest.class)|, where \verb|Category| is in the package \verb|org.junit.experimental.categories| and \verb|IntegrationTest| is in the \verb|maltcms.test| package.

Integration tests are skipped by default, as they may be long-running and resource-consuming. To enable them, add \verb|-DskipITs=false| to your Maven command line, e.g.
\verb|mvn -DskipITs=false install|. Details on the specific configuration for Maltcms can be found in Maltcms' \verb|pom.xml|.

\subsection{Using the Maltcms Testing Infrastructure}\label{sec:using-the-maltcms-testing-infrastructure}
The module \verb|maltcms-test-defaults| contains a number of convenience classes to make testing of Maltcms commands and application a lot easier.
Within the package \verb|maltcms.test|, there are currently three children. \verb|IntegrationTest| is the marker interface used to identify tests that should be run 
by the \verb|maven-failsafe-plugin| mentioned in section \ref{sec:integration-testing}. \verb|SetupLogging| is a \verb|TestWatcher| that can be included in test classes in order to set up \href{http://logging.apache.org/log4j/1.2/}{log4j} compatible logging in conjunction with the \href{http://www.slf4j.org/}{slf4j} binding for \verb|log4j|. To do so, simply create a new member variable:

\begin{lstlisting}[style=JAVA]
@Rule
private final SetupLogging sl = new SetupLogging();
\end{lstlisting}
If you also use \href{http://projectlombok.org/}{lombok}, you can annotate your test case classes with the \verb|@Slf4j| annotation and have a 
readily configured logging system available for your unit tests. To see this in action, have a look at the abstract test base class\verb|AFragmentCommandTest|.
A short excerpt is given here for reference:

\lstinputlisting[style=JAVA,firstnumber=53,firstline=53,lastline=70,float,caption={[Abstract base test class]Abstract base test class showing logging configuration and annotation.},label=lst:afragment-command-test]{../../../maltcms-test-tools/maltcms-test-defaults/src/main/java/maltcms/test/AFragmentCommandTest.java}