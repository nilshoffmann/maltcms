\chapter{Usage}
Maltcms provides a command line interface (CLI) for direct invocation 
from a system's command prompt. Maltcms is usually called by a platform 
dependent wrapper script that is located below Maltcms' installation location:
\begin{itemize}
	\item Microsoft Windows: \verb|bin/maltcms.bat|
	\item Mac OS X: \verb|bin/maltcms.command|
	\item Linux/Unix: \verb|bin/maltcms.sh|
\end{itemize}
These wrapper scripts provide additional options and comfort functions 
over the default Maltcms command line and are thus recommended for general use.

\section{Maltcms Command Line Arguments}
The following basic command line arguments apply to all operating system platforms.

-?                                         display this help
 -a <file1,...>                             alignment anchor files (without path
                                            prefix, assumes location as given
                                            with option -i)
 -b,--createBeanXml <class1,class2, ...>    Creates a fragmentCommands.xml file
                                            in spring format for all instances
                                            of
                                            cross.commands.fragments.AFragmentCo
                                            mmand
 -c <file>                                  configuration file location
 -e <file:///path1/,...>                    extension locations (directories or
                                            .jar files)
 -f <file1,...>                             input files (wildcard, e.g. *.cdf,
                                            file name only if -i is given, or
                                            full path)
 -h                                         display this help
 -i <dir>                                   input base directory
 -l <class1,...>                            print available service providers
                                            (e.g.
                                            cross.commands.fragments.AFragmentCo
                                            mmand)
 -o <dir>                                   target directory for all output
 -p                                         print resolved configuration
 -r                                         recurse into input base directory
 -s,--showProperties <class1,class2, ...>   Prints the properties available for
                                            configuration of given classes

\subsection{Handling Files}

\begin{verbatim}
-r
-i
-f
\end{verbatim}
\subsection{Handling Output}

\paragraph{Output Base-Directory Location}
\begin{verbatim}
-o
\end{verbatim}
\paragraph{Changing the Output Directory Hierarchy}
\begin{verbatim}
-DomitUserTimePrefix
-Doutput.overwrite
\end{verbatim}
\paragraph{Postprocessing of Workflow Results}
selecting and zipping processing results

\subsection{}

\begin{verbatim}
-?                                         display this help
 -a <file1,...>                             alignment anchor files (without path
                                            prefix, assumes location as given
                                            with option -i)
 -b,--createBeanXml <class1,class2, ...>    Creates a fragmentCommands.xml file
                                            in spring format for all instances
                                            of
                                            cross.commands.fragments.AFragmentCo
                                            mmand
 -c <file>                                  configuration file location
 -e <file:///path1/,...>                    extension locations (directories or
                                            .jar files)
 -f <file1,...>                             input files (wildcard, e.g. *.cdf,
                                            file name only if -i is given, or
                                            full path)
 -h                                         display this help
 -i <dir>                                   input base directory
 -l <class1,...>                            print available service providers
                                            (e.g.
                                            cross.commands.fragments.AFragmentCo
                                            mmand)
 -o <dir>                                   target directory for all output
 -p                                         print resolved configuration
 -r                                         recurse into input base directory
 -s,--showProperties <class1,class2, ...>   Prints the properties available for
                                            configuration of given classes
\end{verbatim}

\section{Platform Dependent Wrapper Scripts}

\subsection{MS Windows}
\subsection{MacOS X}
\subsection{Linux/Windows}